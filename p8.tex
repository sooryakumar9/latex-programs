% Numbered Theorems, Definitions, Corollaries

\documentclass{article}

\usepackage{amsthm}
\usepackage{amssymb} % Added for \mathbb{R}

% Define theorem styles
\theoremstyle{plain} % default style for theorems, lemmas, corollaries
\newtheorem{theorem}{Theorem}[section]
\newtheorem{lemma}[theorem]{Lemma}
\newtheorem{corollary}[theorem]{Corollary}
\theoremstyle{definition} % style for definitions
\newtheorem{definition}[theorem]{Definition}

\begin{document}
	
	\section{Introduction}
	This document demonstrates the presentation of numbered theorems, definitions, corollaries, and lemmas.
	
	\begin{theorem}
		Every continuous function defined on a closed interval is bounded and attains its bounds.
	\end{theorem}
	
	\begin{definition}
		A function \( f: [a, b] \to \mathbb{R} \) is called continuous if for every \( x \in [a, b] \) and for every \( \epsilon > 0 \), there exists \( \delta > 0 \) such that whenever \( |x - y| < \delta \), we have \( |f(x) - f(y)| < \epsilon \).
	\end{definition}
	
	\begin{lemma}
		If \( f \) is continuous on a closed interval \( [a, b] \), then \( f \) is uniformly continuous on \( [a, b] \).
	\end{lemma}
	
	\begin{corollary}
		If \( f \) is a continuous function defined on a closed interval \( [a, b] \), then \( f \) is bounded on \( [a, b] \).
	\end{corollary}
	
	
	\section{Examples}
		Here we present some examples to illustrate the theorems and definitions given above.
	
	\begin{theorem}
		The function \( f(x) = x^2 \) is continuous on the interval \([0, 1]\) and attains its maximum and minimum values.
	\end{theorem}
	
	\begin{definition}
		A sequence \( (a_n) \) is called convergent if there exists a number \( L \) such that for every \( \epsilon > 0 \), there exists a positive integer \( N \) such that for all \( n \geq N \), we have \( |a_n - L| < \epsilon \).
	\end{definition}
	
	\begin{lemma}
		Every convergent sequence is bounded.
	\end{lemma}
	
	\begin{corollary}
		If a sequence \( (a_n) \) is convergent, then it is also Cauchy.
	\end{corollary}
	
\end{document}
