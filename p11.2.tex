\documentclass{article}
\usepackage[utf8]{inputenc}
\usepackage{algorithm}
\usepackage{algorithmic}
%\usepackage{algorithm2e}

\title{Algorithm Package}
\author{Abhilash C B}
\date{\today}

\begin{document}
	\maketitle
	
	\section{Question}
	{\Large Develop a LaTeX script to present an algorithm in the document using algorithm/algorithmic/algorithm2e library}
	
	\section*{Illustration}
	\begin{itemize}
		\item \texttt{\textbackslash documentclass\{article\}}: Defines the document type as an article
		\item \texttt{\textbackslash usepackage[utf8]\{inputenc\}}: Ensures proper handling of UTF-8 input characters
		\item \texttt{\textbackslash usepackage\{algorithm\}}: Imports the \texttt{algorithm} package for creating algorithms
		\item \texttt{\textbackslash usepackage\{algorithmic\}}: Imports the \texttt{algorithmic} package for algorithmic pseudo-code
	\end{itemize}
	
	
	Additionally, the following points needs to be remembered: 
	\begin{itemize}
		\item \texttt{\textbackslash begin\{algorithmic\}[1]}: Begins the algorithmic environment with line numbering.
		\item \texttt{\textbackslash begin\{algorithm\}[H]}: Begins the algorithm environment. The \texttt{[H]} option ensures the algorithm is placed exactly where it appears in the text.
	\end{itemize}
	
	\footnote{JSS Academy of Technical Education Bengaluru (CSE)}
	\footnote{JSS Academy of Technical Education Bengaluru (AIML)}
	
	\section*{Training a Machine Learning Model}
	
	\begin{algorithm}[H]
		\caption{Training a Machine Learning Model}
		\begin{algorithmic}[1]
			\REQUIRE Training data $D = \{(x_i, y_i)\}_{i=1}^n$, Learning rate $\eta$, Number of epochs $E$
			\ENSURE Trained model $M$
			\STATE Initialize model parameters $\theta$ randomly
			\FOR{$epoch = 1$ to $E$}
			\FOR{each mini-batch $B \subseteq D$}
			\STATE Compute the gradients $\nabla_\theta L(B, \theta)$
			\STATE Update the model parameters $\theta \leftarrow \theta - \eta \cdot \nabla_\theta L(B, \theta)$
			\ENDFOR
			\STATE Evaluate the model on the validation set
			\IF{validation accuracy has not improved for $k$ epochs}
			\STATE Reduce learning rate $\eta$
			\ENDIF
			\ENDFOR
			\RETURN Trained model $M$
		\end{algorithmic}
	\end{algorithm}
	
	
	
	
	\section*{Find Maximum Value in a List}
	
	\begin{itemize}
		\item \texttt{\textbackslash REQUIRE}: Specifies the input requirements (a list of numbers).
		\item \texttt{\textbackslash ENSURE}: Specifies the output (the maximum value in the list).
		\item \texttt{\textbackslash STATE}: Indicates a single statement within the algorithm.
		\item \texttt{\textbackslash FOR} and \texttt{\textbackslash ENDFOR}: Loop structure.
		\item \texttt{\textbackslash IF} and \texttt{\textbackslash ENDIF}: Conditional structure.
		\item \texttt{\textbackslash RETURN}: Indicates the return value of the algorithm.
	\end{itemize}
	
	\begin{algorithm}
		\caption{Find Maximum Value}
		\begin{algorithmic}[1]
			\REQUIRE A list of numbers $L = [l_1, l_2, \ldots, l_n]$
			\ENSURE The maximum value in the list
			\STATE Let $max \leftarrow L[0]$
			\FOR{each number $x$ in $L$}
			\IF{$x > max$}
			\STATE $max \leftarrow x$
			\ENDIF
			\ENDFOR
			\RETURN $max$
		\end{algorithmic}
	\end{algorithm}
	
	*\section{Algorithm2e Package}
	
	
	\begin{algorithm}[H]
		\caption{Find Maximum Value}
		\KwData{A list of numbers $L = [l_1, l_2, \ldots, l_n]$}
		\KwResult{The maximum value in the list}
		
		$max \leftarrow L[0]$\;
		\For{each number $x$ in $L$}{
			\If{$x > max$}{
				$max \leftarrow x$\;
			}
		}
		\Return $max$\;
	\end{algorithm}
	
\end{document}


