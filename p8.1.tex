\documentclass{article}
\usepackage{amsthm}
\usepackage{amssymb}

\theoremstyle{plain} 
\newtheorem{theorem}{Theorem}[section]
\newtheorem{lemma}[theorem]{Lemma}
\newtheorem{corollary}[theorem]{Corollary}
\theoremstyle{definition} 
\newtheorem{definition}[theorem]{Definition}

\title{Pythagorean Theorem}
\author{JSSATEB}
\date{\today}

\begin{document}
	\maketitle
	
	\section{Introduction}
	The Pythagorean Theorem is a fundamental result in geometry. It states that in a right-angled triangle, the square of the length of the hypotenuse is equal to the sum of the squares of the lengths of the other two sides.
	
	\begin{definition}
		A \textbf{right-angled triangle} is a triangle in which one of the angles is a right angle (\(90^\circ\)).
	\end{definition}
	
	\begin{theorem}[Pythagorean Theorem] 
		In a right-angled triangle, let \( a \) and \( b \) be the lengths of the legs, and let \( c \) be the length of the hypotenuse. Then,
		\[
		a^2 + b^2 = c^2.
		\]
	\end{theorem}
	
	\begin{lemma}
		In an isosceles right-angled triangle, the hypotenuse is \( \sqrt{2} \) times the length of each leg.
	\end{lemma}
	
	\begin{corollary}
		In a right-angled triangle, if the hypotenuse is twice the length of one leg, then the triangle is a 30-60-90 triangle.
	\end{corollary}
	
\end{document}

